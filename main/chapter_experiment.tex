\begin{singlespacing}
\chapter{The ATLAS experiment}
\label{chapter:experiment}
%
\begin{epigraphs}
\qitem{%
if observing outer space gives us a view of the past, observing inner space
would surely give us a glimpse into the future --- would be interesting if NASA
made a telescope for that%
}%
{Ken~M,
\textit{Comment: Yahoo! News},
2012~\cite{kenm2012inner}}
\end{epigraphs}
\end{singlespacing}

Hello


\section{Inner tracker}
\label{sec:atlas_tracker}

\section{Calorimeters}
\label{sec:atlas_calo}

%% Jet / muon start of discussion sand
% Jets are `messy', right?
% Jets do tend to worse resolution --- they suffer from calorimeter noise
% (which varies with statistical properties of hadronic showers
% along with hardware effects), as well as variable
% Jets include neutral hadrons which cannot be tracked, jet clustering may
% incorrectly assign objects, and some hadrons can decay to neutrinos and soft
% muons which would go undetected, and calorimetry is inherently variable.
% This tendency, however, fails for very hard muons.

\section{Muon system}
\label{sec:atlas_muon}

\section{Trigger}
\label{sec:atlas_trigger}
% L1 L2 HLT

\section{Simulation}
\label{sec:atlas_simulation}
% how do we predict data?

\section{Data}
\label{sec:atlas_data}
% discuss collisions, pileup, met, metsig

