\begin{singlespacing}
\chapter{The ATLAS experiment}
\label{chapter:experiment}
%
\begin{epigraphs}
\qitem{%
if observing outer space gives us a view of the past, observing inner space
would surely give us a glimpse into the future --- would be interesting if NASA
made a telescope for that%
}%
{Ken~M,
\textit{Comment: Yahoo! News},
2012~\cite{kenm2012inner}}
\end{epigraphs}
\end{singlespacing}

\TODO{merge into theory chapter?}

from the paper~\cite{atlas2022searches}

\TODO{main detector}
\begin{displayquote}
``%
The ATLAS detector~\cite{PERF-2007-01} is a multipurpose particle detector with
a forward--backward symmetric cylindrical geometry and a near $4\pi$ coverage
in solid angle.%
\footnote{%
``ATLAS uses a right-handed coordinate system with its origin at the nominal
interaction point (IP) in the centre of the detector and the $z$-axis along the
beam pipe.
The $x$-axis points from the IP to the centre of the LHC ring, and the $y$-axis
points upwards.
Cylindrical coordinates $(r,\phi)$ are used in the transverse plane, $\phi$
being the azimuthal angle around the $z$-axis.
The pseudorapidity is defined in terms of the polar angle $\theta$ as
$\eta = -\ln \tan(\theta/2)$.
Angular distance is measured in units of
$\Delta R \equiv \sqrt{(\Delta\eta)^{2} + (\Delta\phi)^{2}}$.%
''~\cite{atlas2022searches}
}
It consists of an inner tracking detector surrounded by a thin superconducting
solenoid providing a $2\,\mathrm{T}$ axial magnetic field, electromagnetic and
hadronic calorimeters, and a muon spectrometer.%
''~\cite{atlas2022searches}
\end{displayquote}


\section{Inner tracker}
\label{sec:atlas_tracker}
\TODO{inner tracker}
\begin{displayquote}
``%
The inner tracking detector covers the pseudorapidity range $|\eta| < 2.5$.
It consists of silicon pixel, silicon microstrip, and transition radiation
tracking detectors.
An additional layer of silicon pixels, the insertable
B-layer~\cite{ATLAS-TDR-19, PIX-2018-001}, was installed before Run~2.%
''~\cite{atlas2022searches}
\end{displayquote}


\section{Calorimeters}
\label{sec:atlas_calo}
\TODO{calo}
\begin{displayquote}
``%
Lead/liquid-argon (LAr) sampling calorimeters provide electromagnetic (EM)
energy measurements with high granularity.
A steel/scintillator-tile hadron calorimeter covers the central pseudorapidity
range ($|\eta| < 1.7$).
The endcap and forward regions are instrumented with LAr calorimeters for both
the EM and hadronic energy measurements up to $|\eta| = 4.9$.%
''~\cite{atlas2022searches}
\end{displayquote}


\section{Muon system}
\label{sec:atlas_muon}
\TODO{muon system}
\begin{displayquote}
``%
The muon spectrometer surrounds the calorimeters and is based on three large
superconducting air-core toroidal magnets with eight coils each.
The field integral of the toroids ranges between $2.0$ and
$6.0\,\mathrm{T\kern-0.15ex m}$ across most of the detector.
The muon spectrometer includes a system of precision chambers for tracking and
fast detectors for triggering.%
''~\cite{atlas2022searches}
\end{displayquote}


\section{Trigger}
\label{sec:atlas_trigger}
% L1 L2 HLT
\TODO{trigger}
\begin{displayquote}
``%
A two-level trigger system is used to select events.
The first-level trigger is implemented in hardware and uses a subset of the
detector information to accept events at a rate below $100\,\mathrm{kHz}$.
This is followed by a software-based trigger that reduces the accepted event
rate to $1\,\mathrm{kHz}$ on average depending on the data-taking conditions.%
''~\cite{atlas2022searches}
\end{displayquote}


\section{Software}
\label{sec:atlas_software}
% how do we predict data?
\TODO{software}
\begin{displayquote}
``%
An extensive software suite~\cite{ATL-SOFT-PUB-2021-001} is used in the
reconstruction and analysis of real and simulated data, in detector operations,
and in the trigger and data acquisition systems of the experiment.%
''~\cite{atlas2022searches}
\end{displayquote}


\section{Data}
\label{sec:atlas_data}
% discuss collisions, pileup, met, metsig

% LHC runs
