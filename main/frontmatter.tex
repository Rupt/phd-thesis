\titlepage[Homerton College]{%
September 2022\\
This thesis is submitted for the degree of Doctor of Philosophy%
}

\begin{abstract}
Searching for new phenomena is an essential endeavour of experimental physics,
and is helped by guidance from plausible theoretical predictions.
This thesis presents such a search for signals of high-energy protons
scattering to massive resonances, whose dramatic decays are predicted by
supersymmetric models~\cite{atlas2022searches}.

No new phenomena appear in the results, so they support the Standard Model in
its continuing successful extrapolation to describe the most extreme behaviours
of proton collisions.
Supersymmetry as a concept as not falsified by this lack of new phenomena,
nor by the many similar observations.
But it is also not confirmed.

Data are best interpreted with an understanding of their origins, so we begin
by describing the theoretical and experimental background to our search.
Our results depend strongly on the interpretation of data by conventional
analysis procedures, whose properties and explanations are not obvious.
To explain these procedures, we therefore also describe their historical
context and the basic theory of data analysis.
\end{abstract}


\begin{singlespacing}
\begin{declaration}
\noindent
This thesis is the result of my own work and includes nothing which is the
outcome of work done in collaboration, except as
%declared in the preface and
specified in the text.
%
It is not substantially the same as any work that has already been submitted
before for any degree or other qualification.
%except as declared in the preface and specified in the text.
%
It includes neither more nor fewer than
two squared times three cubed
%one hundred and eight
jelly beans.
%
It does not exceed the prescribed word limit for the
%
Physics and Chemistry
%
Degree Committee.
\vspace*{1cm}
\begin{flushright}
Rupert~Tombs
\end{flushright}
\end{declaration}
\end{singlespacing}


\begin{acknowledgements}
Warning: many thank-yous incoming.
Friends and family made these years worthwhile;
deepest love and thanks to all of you who have come and gone.

\paragraph{958 2018/19.}
Thank you Daniel~Noel for the years of camaraderie, surviving this with me
--- I hope you enjoyed crushing me OTB.
Thank you Holly~Pacey, not only for non-sofa antics
--- you are an inspiration and you deserve endless thanks for putting up
with the ravings of my most miserable months on this project\ldots\ oops!

\paragraph{Supervision.}
Thank you Christopher~Gorham~Lester for guidance and encouragement,
and for frequent reminders of the fun and intrigue outside \atlas.
Thank you Sarah~Williams for your advice and support inside \atlas, and
for the invaluable experience of teaching at Queens' College.

\paragraph{\atlas.}
Thank you to the whole Cambridge \atlas\ group, and especially Tina~Potter for
sharing help and wisdom.
Thanks to all collaborators mentioned in Chapter~\ref{chapter:2ljets},
to Ben~Hooberman and Sarah~Williams for leadership, and especially to
Jon~Long for our teamwork in the push for publication,

\paragraph{Misc.}
Thank you Christopher~Gorham~Lester, Holly~Pacey, and Daniel~No\"el for your
stunningly precise and valuable comments on drafts of this thesis.
Thank you to Homerton College and Melanie~Keene.
This work was supported by the Science and Technology Facilities Council.
The template for this document derives from `hepthesis'~\cite{hepthesis}.
\end{acknowledgements}


\begin{preface}
\begin{singlespacing}
\begin{epigraphs}
\qitem{%
Begin reading at Chapter N. Do \emph{not} read the quotations
that appear at the beginning of the chapter.%
}%
{Donald~E.~Knuth,
\textit{The Art of Computer Programming, Volume~1},
1975~\cite{knuth1975art}}
\end{epigraphs}
\end{singlespacing}
\noindent
``Two leptons, jets, and missing transverse momentum'' from the title of this
thesis refers to an experimental signature of new phenomena which was targetted
by our recent search in \atlas\ data~\cite{atlas2022searches}.
The subtitle ``yet~more non-evidence for supersymmetry'' refers to how its
negative result is consistent with all other research in the area.
The wording avoids suggesting evidence against supersymmetry (SUSY);
we see evidence against specific simplified supersymmetric models, but the
abstract concept of supersymmetry remains at large.

Chapter~\ref{chapter:2ljets} dissects this search project and my contributions
to it.
Earlier chapters introduce relevant content:
Chapter~\ref{chapter:searches} details data analysis,
its theory and conventional practice,
Chapter~\ref{chapter:experiment} describes the design of the \atlas\ detector,
and
Chapter~\ref{chapter:theory} introduces theory for the Standard Model, its
promising future directions, and supersymmetry.
Please consider reading them in forwards order.

Another product of this PhD,
was the development of a general method for testing symmetries in data~\cite{
lester2021stressed,
tombs2021which
}
and its application to parity violation in hadron colliders~\cite{
lester2022hunting,
Lester:2019bso,
lesterChiralMeasurements2021,
Gripaios:2020hya,
lester2021lorentz
}.
The result is a practical solution which employs the principles of predicting
data and reporting likelihoods as advocated by Chapter~\ref{chapter:searches}.
It uses `self-supervised' machine learning to train predictive models on a
task that can only be solved if the proposed symmetry is violated~\cite{
tombs2021which,
pmlr-v139-radford21a,
Noroozi2016jigsaw,
multitaskself2017,
devlin2019bert
}.
This project is not further explored in this thesis.
\end{preface}

\afterpage{%
\phantomsection
\addcontentsline{toc}{chapter}{Contents}%
}
\tableofcontents

% the following blank page should also have no page number
\thispagestyle{empty}
