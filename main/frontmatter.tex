\titlepage[Homerton College]{%
A dissertation submitted to the University of Cambridge\\
for the degree of Doctor of Philosophy%
}

\begin{abstract}
Searching for new phenomena is an essential endeavour of experimental physics,
and is helped by guidance from plausible theoretical predictions.
This thesis presents such a search for signals of high-energy protons
scattering to massive resonances, whose dramatic decays are predicted by
supersymmetric models~\cite{atlas2022searches}.

Nothing new appears.
Our search finds support the Standard Model in its continuing successful
extrapolation to describe the most extreme behaviours of proton collisions.
Supersymmetry as a concept as not falsified by this lack of new phenomena,
nor by the many similar observations.
But it is also not confirmed.

Data are best interpreted with an understanding of their origins, so we begin
by introducing the theoretical and experimental background to our search, and
the subtleties of our data-analysis procedures in their historical context.
\end{abstract}


\begin{singlespacing}
\begin{declaration}
\noindent
\TODO{
This dissertation is the result of my own work, except where explicit
reference is made to the work of others, and has not been submitted
for another qualification to this or any other university.
This dissertation does not exceed the word limit for the respective
Degree Committee.
}
\vspace*{1cm}
\begin{flushright}
Rupert~Tombs
\end{flushright}
\end{declaration}
\end{singlespacing}


\begin{acknowledgements}
Warning: many thank-yous incoming.
Thank you, thank you, thank you, thank you
to all the friends and family who have come and gone during this project;
I feel much love for you all <3.

From the era of optimal office 958:
thank you to Daniel~Noel for the years of comradery, surviving this with me
--- I hope you enjoyed crushing me over the board;
thank you to Holly~Pacey and not only for antics with the non-sofa ---
you are an inspirational human and academic and I owe you endless thanks for
putting up with the ravings of my most miserable months on this
project\ldots\ oops!

\TODO{supervision}
% Sarah Williams
% for the opportunity to teach at Queens' and valuable experience from
% Christopher Gorham lester

\TODO{analysis team despite our time differences}

\TODO{Cantab}
% Tom Gillam
% Faridka Mustafazade
% Fred Desobry

\TODO{Cambridge: Tina, Richard, working group formerly known as SUSY}

For reviewing drafts of this thesis:
thank you to
Christopher Gorham Lester,
Holly Pacey, and
Daniel No\"el.
Your comments all had stunningly sharp precision and positive impacts on this
thesis and on my morale.

The Science and Technology Facilities Research Council supported this project.
This thesis template uses parts of the \textsc{hepthesis}
class~\cite{hepthesis}.
\end{acknowledgements}


\begin{preface}
\begin{singlespacing}
\begin{epigraphs}
\qitem{%
Begin reading at Chapter N. Do \emph{not} read the quotations
that appear at the beginning of the chapter.%
}%
{Donald~E.~Knuth,
\textit{The Art of Computer Programming, Volume~1},
1975~\cite{knuth1975art}}
\end{epigraphs}
\end{singlespacing}


Data Intensive Science in High-Energy Physics

\TODO{Context of Standard Model theory, directions in which future developments
could improve it, and supersymmetric models}
supersymmery (SUSY)
Chapter~\ref{chapter:theory}

\TODO{the design of the \atlas\ detector}
Chapter~\ref{chapter:experiment}

\TODO{theory and practice of data science, as it could be for clarity
and how it is in our \atlas\ SUSY conventions}
Chapter~\ref{chapter:searches}

\TODO{the main event: the search with many of its gory details in addition
to those presented publically in the paper~\cite{atlas2022searches}}
Chapter~\ref{chapter:2ljets}

\paragraph{Other work.}
\TODO{Use some of the data analysis ideas of Chapter~\ref{chapter:searches}
to build predictive models
and test them to test for symmetries}
\TODO{citations}

\end{preface}

\afterpage{%
\phantomsection
\addcontentsline{toc}{chapter}{Contents}%
}
\tableofcontents

% the following blank page should also have no page number
\thispagestyle{empty}
