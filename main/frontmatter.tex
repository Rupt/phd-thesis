\titlepage[Homerton College]{%
A dissertation submitted to the University of Cambridge\\
for the degree of Doctor of Philosophy%
}

\begin{abstract}
To search for new phenomena is an essential endeavour of experimental physics,
and is enhanced by plausible theoretical predictions.
This thesis presents such a search \atlas\ data for signs of high-energy
protons scattering to massive resonances, whose dramatic decays are
predicted by supersymmetric models~\cite{atlas2022searches}.

Nothing new appears in the results.
This observation supports the Standard Model in its continuing successful
extrapolation to the describe the most extreme behaviours of proton collisions.
Supersymmetry as a concept as not falsified by this lack of visible new
phenomena, nor by the many similar observations.
But it is also not confirmed.

Data are best interpreted with an understanding of their background, so we
begin by introducing the theoretical and experimental background to our search,
and the subtleties of its data-analysis procedures in their historical context.
\end{abstract}


\begin{declaration}
\noindent
This dissertation is the result of my own work, except where explicit
reference is made to the work of others, and has not been submitted
for another qualification to this or any other university.
This dissertation does not exceed the word limit for the respective
Degree Committee.
\vspace*{1cm}
\begin{flushright}
Rupert~Tombs
\end{flushright}
\end{declaration}


\begin{acknowledgements}
<3
% Holly
% Dan
% Chris

% Friends who have come and gone from Homerton over the years

% Sarah Williams
% Ben Hooberman
% Jon Long

% Family, alive and dead.

This document uses parts of the ``hepthesis'' class~\cite{hepthesis}.
\end{acknowledgements}


\begin{preface}
\begin{singlespacing}
\begin{epigraphs}
\qitem{%
Begin reading at Chapter N. Do \emph{not} read the quotations
that appear at the beginning of the chapter.%
}%
{Donald~E.~Knuth,
\textit{The Art of Computer Programming, Volume~1},
1975~\cite{knuth1975art}}
\end{epigraphs}
\end{singlespacing}


Data Intensive Science in High-Energy Physics

\TODO{Context of Standard Model theory, directions in which future developments
could improve it, and supersymmetric models}
supersymmery (SUSY)
Chapter~\ref{chapter:theory}

\TODO{the design of the \atlas\ detector}
Chapter~\ref{chapter:experiment}

\TODO{theory and practice of data science, as it could be for clarity
and how it is in our \atlas\ SUSY conventions}
Chapter~\ref{chapter:searches}

\TODO{the main event: the search with many of its gory details in addition
to those presented publically in the paper~\cite{atlas2022searches}}
Chapter~\ref{chapter:2ljets}

\paragraph{Other work.}
\TODO{Use some of the data analysis ideas of Chapter~\ref{chapter:searches}
to build predictive models
and test them to test for symmetries}
\TODO{citations}

\end{preface}

\afterpage{%
\phantomsection
\addcontentsline{toc}{chapter}{Contents}%
}
\tableofcontents

% the following blank page should also have no page number
\thispagestyle{empty}
