\setcounter{chapter}{-1}
\begin{singlespacing}
\chapter{Introduction}
\label{chapter:introduction}
%
\begin{epigraphs}
\qitem{%
\TODO{}}%
{\TODO{},
\textit{\TODO{}},
\TODO{}}
\end{epigraphs}
\end{singlespacing}

Data Intensive Science in High-Energy Physics

\section{Other work}
\TODO{This does not belong in an introduction and will be moved.}


\subsection{Testing symmetries in data}
Data-driven tests of physical symmetries are a speciality of the Cambridge
\atlas\ group.
Each test considers a theoretically motivated symmetry that states that
distributions of data should be identical when transformed under its operation.
This means that it can be possible to disprove a symmetry by searching only
for differences between data and transformed versions of those data, without
necessarily requiring other theoretical inputs.

% emus
Charge symmetry of electron-muon, `emu', pairs is a key example among these
tests.
Theoretically, it proposes that, in the absence of biases, data
from proton-proton collisions at the LHC would include equal distributions
of $e^+\mu^-$ and $e^-\mu^+$ events~\cite{Lester:2016qdv}.
Real data can have emu biases --- one clear (but minor) example is that our
detectors are made of matter containing negatively charged electrons, not
positrons, so high energy electrons positrons have subtly different rates of
energy loss in that detector matter.
However, \cite{Lester:2016qdv} argues that any biases at the LHC should
conspire only to favour the rate of $e^-\mu^+$ over the rate of $e^+\mu^-$
events.
Therefore defining an asymmetry
\begin{equation}
A^\mathrm{emu} = \frac{
\sigma(e^+\mu^- + X)
}{
\sigma(e^-\mu^+ + X)
}
,
\end{equation}
one can predict $A^\mathrm{emu} \leq 1$ from the Standard Model at the LHC.
Negative values could be explained by experimental biases, but
$A^\mathrm{emu} > 1$ measured in data might indicate new physics, since it and
can be caused by models of new physics that might invoke R-parity violating
supersymmetry~\cite{Lester:2016qdv} or leptoquarks~\cite{EXOT-2018-29}.
After its extensive development~\cite{Brunt:2674708,Pacey:2747774},
this test has been performed in \atlas\ data, and found no ``significant
model-independent evidence''~\cite{EXOT-2018-29} for $A^\mathrm{emu} > 1$.

% parity violation
Parity symmetry, which changes the signs of all spatial coordinates
$\vec x\rightarrow -\vec x$, is a second key example.
Nature does violate parity symmetry, as first shown by the Wu
experiment~\cite{PhysRev.105.1413}, in the weak sector, but that is not
directly observable at the LHC due to its non-polarized beams and
polarization-insensitive detectors~\cite{Lester:2019bso}.
Unknown physical effects could, however, generate forms of parity asymmetry
that could be seen at the LHC.
Although the charge symmetry of the emu test has known sources of bias,
no such biases are so clearly expected for parity.

General parity symmetry can therefore be tested at the LHC by considering a
parity-odd event variable and comparing its distribution to its parity-flipped
version, which simply swaps its sign.
% alpha from lester schott
One such test has been performed with LHC data~\cite{Lester:2019bso}.
This test used \cms\ open data and the parity-odd event variable
\begin{equation}
\label{eqn:cms_alpha}
\alpha
= \arcsin\!
\left(
\frac{
\vec p^{\,j_1} \times \vec p^{\,j_2}
}{
|\vec p^{\,j_1} \times \vec p^{\,j_2}|
}
\cdot
\frac{
\vec p^{\,j_3}
}{
|\vec p^{\,j_3}|
}
\right)
,
\end{equation}
in which $p^{\,j_i}$ is the momentum of the $i$th hardest jet in a multi-jet
event,
and did not find evidence for parity violation.
But that does not prove that no parity-violating effects are occuring in these
data, only that it did not appear in the distribution of $\alpha$.

Can a single parity-odd event variable be sensitive to all norms of parity
violation?
The answer, it turns out, is `no' if that variable is also required to be
continuous~\cite{lesterChiralMeasurements2021}.
In this framework, therefore, a joint analysis of multiple variables is
required for any sufficiently complicated data.
Furthermore, it is not generally easy to construct such universally-sensitive
sets of event variables~\cite{
Gripaios:2020hya,
Lester:2020jrg
}.

Rather than beginning from universally sensitive event variables, an
alternative strategy is to build candidate variables to the data
themselves~\cite{lester2021stressed}.
If done wrong, such an idea could simply reinforce variations due to simple
statistical noise.
It can, however, be formulated as a rigorous test if one employs standard
Machine Learning practice and splits the data into independent training and
testing sets.
Variables learned (by function approximation algorithms such as
Boosted Decision Trees~\cite{xgboost} or
Neural Networks~\cite{MurphyKevinP.2012Mlap})
from the testing set can correspond to hypotheses about possible asymmetries
in the data distribution, and if those hypotheses make accurate predictions in
independent testing data, then they are evidence against the symmetry.

My contribution to this field is to adapt this idea of learning parity
variables,
\emph{which was initiated by Christopher~G.~Lester}~\cite{lester2021stressed},
into a probabilistic description, in which the algorithms predict
`which is real?' between real data an their parity-transformed
copies~\cite{tombs2021which}.
This approach derives an objective function
(which is a special case of the binary
cross-entropy~\cite{MurphyKevinP.2012Mlap}),
from which learning algorithms
can approximate the odds ratio between the data distribution and its
parity-transformed version.
Importantly, it also allows one to model inefficiencies or holes in the
distribution, which exist in practice from kinematic selections and imperfect
hardware, and incorporate those into the training and testing without biasing
the result.
We have also demonstrated this method on simulated multi-jet data in an
\atlas-like detector~\cite{lester2022hunting}.
By simulating parity-violating events from a Lorentz-violating extension
to the Standard Model, we show that this method build variables that are much
more sensitive than the $\alpha$ of Equation~\ref{eqn:cms_alpha} while
retaining its other attractive symmetries by encoding them into our function
architectures~\cite{lester2021stressed, tombs2021which}.

This work demonstrates that the `which is real?' method is ready to test for
parity violation in real LHC data.
Much like the emu work, however, its realization would require great diligence
to handle important considerations of systematic uncertainties and statistical
interpretations before it could be published through \atlas.

The `which is real?' method is general, and can test not only parity, but
symmetry under any transformation of the data, irrespective of whether it is
discrete or continuous.
In the context of modern machine learning, it is an example of
`self-supervised' learning, in which the training objective derives from the
data and not from additional labels~\cite{
multitaskself2017,
devlin2019bert
}, and quite similar to the `CLIP' objective, which also attempts to identify
a real examples from collections of alternatives~\cite{pmlr-v139-radford21a}.
What is special in our work is our application for scientific self-supervised
testing.

% this comment activates enlarged spacing for the final paragraph lol
