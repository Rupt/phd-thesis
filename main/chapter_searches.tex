\begin{singlespacing}
\chapter{Searches in ATLAS}
\label{chapter:intro}
%
\begin{epigraphs}
\qitem{%
One way is to make it so simple that there are \emph{obviously} no deficiencies
and the other way is to make it so complicated that there are no \emph{obvious}
deficiencies.%
}%
{Tony~Hoare,
\textit{The Emperor’s Old Clothes},
1981~\cite{hoare2007emperor}}
\end{epigraphs}
\end{singlespacing}


\section{Searches}
\label{sec:searches_searches}
% histograms, CR VR SR, systematics

\section{Data analysis}

\section{Frequentist theory}
\begin{epigraphs}
\qitem{%
Within this theory, statistical methods of great practical usefulness have been
developed, and its statements can and frequently do contribute in a vague way
to the interpretation of data. \ldots%
}%
{John~W.~Pratt,
\textit{Review: Testing Statistical Hypotheses},
1961~\cite{pratt1961testing}}
\end{epigraphs}

\section{Post-frequentist practice}
\begin{epigraphs}
\qitem{%
\ldots\ But this book, by its very excellence, its thoroughness, lucidity and
precision, intensifies my growing feeling that nevertheless the theory is
arbitrary, be it however ``objective,'' and the problems it solves, however
precisely it may solve them, are not even simplified theoretical counterparts
of the real problems to which it is applied.%
}%
{John~W.~Pratt,
\textit{Review: Testing Statistical Hypotheses},
1961~\cite{pratt1961testing}}
\end{epigraphs}

% significancen measures
The ATLAS-recommended significance measure~\cite{atlas_significance} is
\begin{align}
\label{eqn:significance_atlas}
S_\mathrm{\atlas}(n; \mu, \sigma) =~&
\mathrm{sgn}(n - \mu) \times \sqrt{2} \\[0.2em] \nonumber
&
\times \sqrt{
n\log\left(\frac{n(\mu + \sigma^2)}{\mu^2 + n\sigma^2}\right)
- \frac{\mu^2}{\sigma^2}\log\left(
1 + \frac{\sigma^2(n - \mu)}{\mu(\mu + \sigma^2)}
\right)
}
.
\end{align}


\clearpage

Hello again
