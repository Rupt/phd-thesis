\begin{singlespacing}
\chapter{The Standard Model and beyond}
\label{chapter:theory}
%
\begin{epigraphs}
\qitem{%
The idea that theorems follow from the postulates does not correspond to simple
observation.
If the Pythagorean theorem were found to not follow from the postulates, we
would again search for a way to alter the postulates until it was true.
Euclid's postulates came from the Pythagorean theorem, not the other way.}%
{Richard~Hamming,
\textit{The Unreasonable Effectiveness of Mathematics},
1980~\cite{hamming1980unreasonable}}
\end{epigraphs}
\end{singlespacing}
\noindent
The Standard Model encodes out best understanding of nature's fundamental
objects in their behaviours as particles.
From solid mathematical foundations, it accurately describe the physics
of particle interactions across a broad domain,
but that domain is bounded, so the Standard Model is not an ultimate theory of
everything.
Work continues to expand that boundary on both theoretical and experimental
fronts.
The current Standard Model is described in Section~\ref{sec:theory_sm},
along with its limitations and possible future directions.

Supersymmetry an idea that has been applied to build plausible alternatives
to the Standard Model which may explain future experimental discoveries.
Since the main work of this thesis is a search for effects of supersymmetric
phenomena, supersymmetry and the specific supersymmetric models that we study
are presented in Section~\ref{sec:theory_susy}.


\section{The Standard Model}
\label{sec:theory_sm}
The current Standard Model features fermions
(of four types in three generations),
which interact through forces mediated by vector bosons
and the scalar Higgs boson.
The strong force is mediated by
Massless gluons $g$ mediate the strong force,
the massless photon $\gamma$ mediates the electromagnetic force,
and very massive $W^+$, $W^-$, and $Z$ bosons mediate the weak force.
These fundamental particles of the Standard Model are summarized in
Table~\ref{tab:theory_particles_sm}.

As a gauge field theory, the Standard Model has the symmetry group
$SU\!(3)_C \times SU\!(2)_L \times U\!(1)$,
meaning that its lagrangian is invariant under local operations of that group
on its fields in certain representations.

Gluons in this model correspond to the eight generators of $SU\!(3)_C$, whose
subscript $_C$ indicates that they operate on colour-charge triplets of quarks.

Electromagnetic and weak forces are unified into the electroweak
interactions with four bosons $W^1$, $W^2$, $W^3$, and $B$ from the
$SU\!(2)_L$ and $U\!(1)$ symmetries respectively.
These would be massless if the symmetry were respected, but it is
`spontaneously' broken with the Higgs mechanism~\cite{
higgs1964broken,
englert1964broken
}%
\footnote{%
Historical accuracy prefers the name `Brout–Englert–Higgs mechanism', but
Peter~Higgs has reportedly suggested
``ABEGHHKtH mechanism''~\cite{close2011infinity}
to fully credit other associated work.%
}
which adds four scalar fields (as a complex doublet) with a potential that
moves the ground state to a gauge-\emph{asymmetric} configuration.
Around this new ground state, linear combinations of the electroweak fields
acquire masses while ``eating up''~\cite{rubakov1999classical} three of the
four scalar fields to yield the three weak bosons and the photon:
$W^1$ and $W^2$ mix into $W^+$ and $W^-$, while $W^3$ and $B$ mix into $Z$ and
$\gamma$.
The fourth scalar is the Higgs boson $h$~\cite{
glashow1959renorm,
weinberg1967model,
salam1959weak,
rubakov1999classical,
cottingham2007greenwood
}.

Fermion masses arise through their Yukawa interactions with the Higgs field.
The subscript $_L$ on $SU\!(2)_L$ indicates that the weak interaction only
couples to fermions in left-chiral doublets: $(\nu_e, e)_L$,
$(u, d)_L$ and their repeats across the three generations.
In each generational trio (such as $u$, $c$, $t$), all three fermions share the
same quantum numbers, but have different masses.
A second generational difference occurs in quarks and neutrinos through the
difference between their weak flavour eigenstates and the mass eigenstates of
their free propagation;
flavour and mass eigenstates are linearly related through unitary matrices:
the CKM matrix for the quarks~\cite{
cabibbo1963unitary,
kabayasji1973cpv
}
and the PMNS matrix for the neutrinos~\cite{
maki1962remarks,
thomson2013modern
}.

% using https://www.tablesgenerator.com/latex_tables
\begin{table}[tp]
\centering
\begin{tabular}{rrc}
\multirow{4}{*}{fermions} & \multirow{2}{*}{quarks}      & \makebox[5em]{\hfill $u$ \hfill $c$ \hfill $t$ \hfill}                    \\
                          &                              & \makebox[5em]{\hfill $d$ \hfill $s$ \hfill $b$ \hfill}                    \\[1ex]
                          & \multirow{2}{*}{leptions}    & \makebox[5em]{\hfill $e$ \hfill $\mu$ \hfill $\tau$ \hfill}               \\
                          &                              & \makebox[5em]{\hfill $\nu_e$ \hfill $\nu_\mu$ \hfill $\nu_\tau$ \hfill} \\[3ex]
\multirow{2}{*}{bosons}   & strong                       & \makebox[7em]{\hfill $g$ \hfill}                                          \\[1ex]
                          & electroweak & \makebox[7em]{\hfill $W^\pm\!\!$ \hfill $Z$\hfill $\gamma$ \hfill $h$ \hfill} \\
\end{tabular}
\caption[Particles of the Standard Model]{%
Particles of the Standard Model.
All fermions are spin-$1/2$, and all bosons are spin-$1$ except for the
spin-$0$ Higgs boson $h$.
Fermions come in left- and right-chiralities, although the existence of
right-chiral neutrinos is not strictly confirmed.%
}
\label{tab:theory_particles_sm}
\end{table}

Early versions of the Standard Model had only two fermion generations and
no neutrino masses~\cite{
wells2020discovery,
bjorken1985november
},
but the third generation and non-zero neutrino masses are now observed;
for example, the third generation is shown by charge-parity $CP$ asymmetry in
kaons~\cite{
cronin1964evidence,
kabayasji1973cpv
},
precise measurements of the $Z$ boson~\cite{
lep2006precision
},
and direct observations~\cite{
perl1977evidence,
herb1977observation,
abachi1995observation
};
neutrino masses are also necessary to explain their observed flavour
oscillations~\cite{
kamiokande1998measurement,
superk1998evidence,
superk1999measurement,
lsnd1998evidence
}.

Various theories can describe neutrino masses in the Standard Model, and it is
currently uncertain which form best describes nature;
this is both a limitation of the current Standard Model and an
opportunity for future refinement.
With future results, the Standard Model will continue to improve in its
correspondence with nature, and we discuss some such opportunities
for improvement in the next section.


\section{Opportunities for future Standard Models}
The Standard Model works very well within its domain.
This discussion therefore looks towards where an altered Standard Model could
improve its current abilities or to expand its generality to describe more of
the universe.

When `nobody knows' the answer to a riddle, the status lies somewhere between
two extremes:
either many answers are satisfactory with no clear best,
or no satisfactory answers are yet known.
We distinguish these two cases with the labels
\emph{uncertain} and \emph{unknown} respectively:
the model of neutrino masses is mostly \emph{uncertain}, but
the reconciliation quantum field theory with general relativity is
mostly \emph{unknown}.

One real problem with the Standard Model is that algorithms to compute its
general consequences are unknown, since practical perturbative and discrete
lattice methods are limited.
Imprecise modelling weakens experimental results (anomalous or not),
but does not present an opportunity because it is common to the quantum field
theory that is used by all known viable models.
This section briefly reviews some other directions of positive change for
future models.

\subsection{Neutrino masses}
Neutrinos could be Dirac fermions with Yukawa masses and boring right-chiral
fields.
They could also have Majorana mass terms, for which the seesaw mechanism can
simultaneously generate the observed light neutrinos with very heavy `sterile'
neutrinos.
Or their masses could come from other operators necessarily introducing the
right-chiral fields~\cite{
thomson2013modern,
chala2021neutrino,
wells2020discovery
}.
The answer is uncertain, and new results from theory or experiment that
pinpoint a correct model could improve the Standard Model in the neutrino
domain.

\subsection{Gravitation and cosmology}
Gravity does not feature in the Standard Model, nor does it measurably impact
particle physics experiments.
Although gravitation is in another domain, the particle content of the
Standard Model does impact cosmology.

The universe contains dark matter which interacts with gravity but not much
with other forces; it is observed through galactic rotation curves,
gravitational lensing, and cosmic microwave background data,
but its composition is uncertain~\cite{
begeman1991rotation,
garrett2010dark,
planck2020results
}.
Dark matter candidates include new particles such as
sterile neutrinos~\cite{boyarsky2019sterile},
Weakly Interacting Massive Particles (WIMPs)~\cite{jungman1996wimp},
or
axion-like particles~\cite{kim2008axions},
but also macroscopic objects such as
primordial black holes~\cite{bernard2019primordial}.
Dark matter may well comprise various of different components, and it would
be a success for future particle physics theories to pinpoint necessary dark
matter particles.

Other mysteries of cosmology include:
the small positive cosmological constant,
the inflation of the early universe,
and baryogenesis~\cite{
wells2020discovery,
riess1998observational
}.
Solutions to these mysteries all lie somewhere between uncertain and unknown,
but may not be entirely out of reach;
for example, the Higgs field could plausibly have driven
inflation~\cite{bezrukov2008higgs},
and the Standard Model does slightly violate $C$ and $CP$, both of which
are required for baryogenesis~\cite{sakharov1991re}.

\subsection{Origins}
Why does the Standard Model have its structure?
Why is it configured with the observed values of our parameters?
Why do we even live in $3+1$ dimensional spacetime with quantum mechanics?
Answers to these questions are largely unknown, but would be
attractive features for a fundamental theory.

Certain Standard Model structures are quite general, and are parametrized such
that parts can be smoothly disabled if they do not describe nature.
For example, the strong interaction includes a parameter $\theta_\mathrm{CP}$
whose non-zero values would induce $\mathrm{CP}$ violation in the strong
sector.
Observations find that $\theta_\mathrm{CP}$ cannot be far from zero, and one
attractive feature of axion-like models is their requirement that
$\theta_\mathrm{CP} = 0$~\cite{
kim2008axions,
thomson2013modern,
martin2017particle
}.

Justified constraints on the model could have led to better predictions of
past data and might extrapolate to new phenomena through their deeper
understanding of nature.
Attempts to relate empirical parameters~\cite{
sato1979ratio,
beg1982gauge,
koide1983fermion,
denterria2012gaussian
}
or to derive mathematical structures~\cite{
goyal2010origin,
skilling2021abc,
axioms1010038
}
are curious, but will remain speculative until they provide novel benefits.

\subsection{Aesthetics}
In the search for improved models, heuristic criteria can help to identify
appealing candidates.
Two such criteria are `fine-tuning' and `naturalness', both of which are
aesthetic judgements about dimensionless model parameters.
Naturalness is opinion; only nature decides what is really natural,
but these criteria can still guide the discourse of our scientific searches
towards useful directions~\cite{wells2020discovery}.

Fine-tuning criteria dislike tight constraints, and naturalness contains
various ideas of which one is that nice numbers should have moderate orders of
magnitude~\cite{giudice2008naturally}.
For example, $\theta_\mathrm{CP}$ is unattractively fine-tuned by an
observation of $\theta_\mathrm{CP} \lesssim 10^{-11}$, and that
large $11$ in its exponent also defies naturalness.

Another example is the `hierarchy' problem of the Higgs boson mass $m(h)$,
which notes that new particles at a heavy mass scale $\Lambda$ induce changes
of $\Delta m(h)^2 = \mathcal{O}(\Lambda^2)$,
through direct or indirect couplings.
With any particles near the Planck scale $\Lambda \sim 10^{18}\,\eV[G]$,
implementing the observed $m(h) \sim 10^{2}\,\eV[G]$ therefore requires
finely-tuned cancellations between positive and negative contributions.
New fermions subtract from $m(h)$ and new scalars add by similar amounts,
so new models can implement this fine-tuning by inserting only matched sets of
fermions and bosons.
This solution is the strategy of supersymmetric models.


\section{Supersymmetric models}
\label{sec:theory_susy}
Supersymmetry is the concept of symmetry under the operation of exchanging
bosonic and fermionic states, and supersymmetric models use supersymmetry in
their construction.
Nature is not totally supersymmetric.
If it were, then there would be bosonic `superpartners' for each of the known
fermions (and vice versa) with the same masses and charges, and those
`sparticles' would have been seen long ago.
But supersymmetry might still participate in a broken condition if the
sparticles are hidden at larger masses.

Any new model must include the observational facts of the Standard Model,
and the Minimal Supersymmetric Standard Model (MSSM) satisfies this while
adding a school of new particles,
which is listed in Table~\ref{tab:theory_particles_mssm}.

The MSSM matches each standard fermion to a pair of scalar
`sfermions' (left and right)
and each vector gauge field to a `gaugino' fermion
($\tilde g$, $\tilde W^1$, $\tilde W^2$, $\tilde W^3$, and $\tilde B$).
The Higgs field does not translate so directly, but is replaced with two sets
of complex scalar fields
($H^+_u$, $H^0_u$ and $H^0_d$, $H^\pm_d$) with
gaugino superpartners
($\tilde H^+_u$, $\tilde H^0_u$ and $\tilde H^0_d$, $\tilde H^-_d$).
(All sparticles are marked with tildes $\tilde{~}$.)
These fields mix into mass eigenstates which current data do not entirely
determine:
the gaugino mixtures are the charged charginos $C^\pm_i$ and
neutral neutralinos $N_i$
(also labelled $\chargino_i$ and $\neutralino_i$~\cite{atlas2022searches}),
and each squark and slepton has two states that mix the left and right fields.
From the four complex scalars, three degrees of freedom are eaten by the
weak bosons to leave the reborn Standard Model Higgs boson $h$
and four new scalars~\cite{
martin2016primer
}.

Current observations do limit the masses and mixings of this new
school of MSSM particles, but their values remain very uncertain and in need
of new experimental constraints.

% made using:
% https://www.tablesgenerator.com/latex_tables
\begin{table}[tp]
\centering
\begin{tabular}{rrc}
\multirow{5}{*}{bosons}   & Higgs                         & \makebox[7em]{\hfill ($h$) \hfill $H^0$ \hfill $A^0$ \hfill $H^\pm\!\!$ \hfill}             \\[1ex]
                          & \multirow{2}{*}{squarks}      & \makebox[5em]{\hfill $\tilde u$ \hfill $\tilde c$ \hfill $\tilde t$ \hfill}                  \\
                          &                               & \makebox[5em]{\hfill $\tilde d$ \hfill $\tilde s$ \hfill $\tilde b$ \hfill}                  \\[1ex]
                          & \multirow{2}{*}{sleptions}    & \makebox[5em]{\hfill $\tilde e$ \hfill $\tilde \mu$ \hfill $\tilde \tau$ \hfill}             \\
                          &                               & \makebox[5em]{\hfill $\tilde \nu_e$ \hfill $\tilde \nu_\mu$ \hfill $\tilde \nu_\tau$ \hfill} \\[3ex]
\multirow{3}{*}{fermions} & gluinos                       & \makebox[5em]{\hfill $\tilde g$ \hfill}                                        \\[1ex]
                          & \multirow{2}{*}{gauginos}     & \makebox[5em]{\hfill $\tilde C^\pm_1\!\!$ \hfill $\tilde C^\pm_2\!\!$ \hfill} \\
                          &                               & \makebox[7em]{\hfill $\tilde N_1\!$ \hfill $\tilde N_2\!$ \hfill$\tilde N_3\!$ \hfill$\tilde N_4\!$ \hfill} \\
\end{tabular}
\caption[%
Particles of the Minimal Supersymmetric Standard Model%
]{%
Particles of the Minimal Supersymmetric Standard Model
in their mass eigenstates~\cite{martin2016primer}, in addition to the
Standard Model particles of Table~\ref{tab:theory_particles_sm}.
The standard Higgs boson ($h$) is repeated here with its new pals.
Sparticles are marked with tildes $\tilde{~}$.
All listed bosons are spin-$0$ and all listed fermions are spin-$1/2$.
}
\label{tab:theory_particles_mssm}
\end{table}

Various mechanisms could spontaneously break supersymmetry.
Gravitational ideas exist, but one model considered in this thesis uses a
Gauge-Mediated Supersymmetry Breaking (GMSB) mechanism,
which adds yet more elusive fields:
a `secluded' sector which breaks the symmetry, and
a `messenger' sector which
communicates that breaking effect back to the observable sector.
Such models also add a spin-$3/2$ `gravitino' ($\gravitino$),
which should be light and whose helicity $\pm1/2$ components would interact
in collider physics~\cite{
giudice1999gmsb,
dine1982pheno,
martin2016primer
}.

All supersymmetric models considered in this thesis conserve $R$-parity
\begin{equation}
P_R = (-1)^{3(B - L) + 2s}
\end{equation}
(with baryon number $B$, lepton number $L$ and spin $s$)
in their interactions~\cite{
farrar1978rparity,
martin2016primer
}.
Conservation of $R$-parity helps by preventing proton decays,
but also makes the Lightest Supersymmetric Particle (LSP) a plausible
WIMP dark matter candidate since it cannot decay to purely Standard Model
particles.
Providing this LSP dark matter candidate through a production mechanism
that could be observable at the LHC is a strong motivating factor
both for the supersymmetric models
and for the experimental searches of this thesis.

Supersymmetric modelling successfully predicted the lightest Higgs boson
mass to be $\lesssim 130\,\eV[G]$ long before its observation at
$125\,\eV[G]$~\cite{
espinosa1992higgs,
espinosa1993upper,
quiros1997constraints,
wells2020discovery,
atlas2012higgs,
atlas2012combined,
cms2012higgs
}.
This prediction, however, assumed a mass scale below $1\,\eV[T]$,
and observations at the LHC are now limiting sparticle masses
into super-$\eV[T]$ territory (in specific models).
Such limits are also troublesome for the hierarchy problem, because large
superpartner masses introduce fine-tuning problems of their
own~\cite{martin2016primer}.
Despite its current lack of observational evidence, however,
supersymmetry remains a tool for building theoretically plausible
models, and might help us to understand future discoveries.

% this comment activates enlarged spacing for the final paragraph lol
